%%%%%%%%%%%%%%%%%%%%%%%%%%%%%%%%%%%%%%%%%%%%%%%%%%%%%%%%%%%%%%%%%%%%%%%%%%%%%%
% Vorlage für Ihre Lösungen
%%%%%%%%%%%%%%%%%%%%%%%%%%%%%%%%%%%%%%%%%%%%%%%%%%%%%%%%%%%%%%%%%%%%%%%%%%%%%%

\documentclass[11pt,a4paper,DIV=12]{scrartcl}

% Pakete einbinden, die benötigt werden
\usepackage{scrlayer-scrpage}
\usepackage[utf8]{inputenc}       % Dateien in UTF-8 benutzen
\usepackage[T1]{fontenc}          % Zeichenkodierung
\usepackage{graphicx}             % Bilder einbinden
\usepackage[main=ngerman, english]{babel}  % Deutsche Sprachunterstützung
\usepackage[autostyle=true,german=quotes]{csquotes} % Deutsche Anführungszeichen
\usepackage[pagebackref=false,german]{hyperref} % Hyperlinks
\usepackage{xcolor}               % Unterstützung für Farben
\usepackage{amsmath}              % Mathematische Formeln
\usepackage{amsfonts}             % Mathematische Zeichensätze
\usepackage{amssymb}              % Mathematische Symbole
\usepackage{float}                % Fließende Objekte (Tabellen, Grafiken etc.)
\usepackage{booktabs}             % Korrekter Tabellensatz
\usepackage{listings}             % Quelltexte
\usepackage{listingsutf8}         % Quelltexte in UTF8
\usepackage[hang,font={sf,footnotesize},labelfont={footnotesize,bf}]{caption} % Beschriftungen
\usepackage[scaled]{helvet}       % Schrift Helvetia laden
\usepackage[bottom=25mm,left=30mm,right=30mm,top=25mm]{geometry} % Ränder ändern
\usepackage{setspace}             % Abstände korrigieren
\usepackage{scrhack}              % tocbasic Warnung entfernen
\usepackage[all]{hypcap}          % Korrekte Verlinkung von Floats
\usepackage{tabularx}             % Spezielle Tabellen
\usepackage{rotating}             % Seiten drehen
\usepackage{array}

 % Einstellungen für Schriftarten
\setkomafont{titlehead}{\centering\normalfont\sffamily}
\setkomafont{author}{\normalfont\sffamily}
\setkomafont{publishers}{\normalfont\sffamily}
\setkomafont{date}{\normalfont\sffamily}
\setkomafont{title}{\normalfont\sffamily\bfseries\LARGE}
\setkomafont{pagehead}{\normalfont\sffamily}
\setkomafont{pagenumber}{\normalfont\sffamily}
\setkomafont{paragraph}{\sffamily\bfseries\small}
\setkomafont{subsection}{\sffamily\bfseries\Large}
\setkomafont{subsubsection}{\sffamily\itshape\bfseries\small}
\addtokomafont{footnote}{\footnotesize}

% Wichtige Abstände
\setlength{\parskip}{0.2cm}  % 2mm Abstand zwischen zwei Absätzen
\setlength{\parindent}{0mm}  % Absätze nicht einziehen
\clubpenalty = 10000         % Keine "Schusterjungen"
\widowpenalty = 10000        % Keine "Hurenkinder"
\displaywidowpenalty = 10000 % Keine "Hurenkinder"
                             % Siehe: https://de.wikipedia.org/wiki/Hurenkind_und_Schusterjunge

% Einfacher Font-Wechsel über dieses Makro
\newcommand{\changefont}[3]{
\fontfamily{#1} \fontseries{#2} \fontshape{#3} \selectfont}

\changefont{ptm}{m}{n}  % Times New Roman für den Fließtext
\renewcommand{\rmdefault}{ptm}

\newcommand{\loesung}{\textbf{Lösung:}\\}

\newcommand{\kapitel}[1]{\setcounter{section}{#1}}

\newcommand{\kapitelname}[1]{\title{Übungsblatt \thesection: #1}}

\newcolumntype{L}{>{\raggedright\arraybackslash}X}

\newcolumntype{B}[1]{r*{#1}{@{\,}r}}

\RedeclareSectionCommand[
  beforeskip=-1.0\baselineskip,
  afterskip=0.01\baselineskip]{section}

\RedeclareSectionCommand[
  beforeskip=-1.0\baselineskip,
  afterskip=0.01\baselineskip]{subsection}

\titlehead{Webbasierte Systeme (WEB) -- Prof. Michael Gröschel -- Sommersemester 2020}

\date{\today}

%%%%%%%%%%%%%%%%%%%%%%%%%%%%%%%%%%%%%%%%%%%%%%%%%%%%%%%%%%%%%%%%%%%%%%%%%%%%%%
% Hier beginnt das eigentliche Dokument
% An den Einstellungen davor müssen Sie normalerweise keine Änderungen
% vornehmen.
%%%%%%%%%%%%%%%%%%%%%%%%%%%%%%%%%%%%%%%%%%%%%%%%%%%%%%%%%%%%%%%%%%%%%%%%%%%%%%

% Bitte tragen Sie hier Ihren Namen und Ihre Mattrikelnummer ein
\author{Aleksandar Hristov\\ (1722385)}

% Welches Übungsblatt zu welchem Thema bearbeiten Sie?
\kapitel{1}
\kapitelname{Grundlagen und Architektur des WWW}

\begin{document} % Hier beginnt das eigentliche Dokument
\maketitle       % Titel erzeugen

% ----------------------------------------------------------------------------
% Jede Aufgabe bekommt eine subsection. Die Nummerierung erfolgt automatisch
\subsection{Auffrischung}
Erklären Sie die folgenden Grundbegriffe:
\begin{itemize}
    \item Socket 
    \item Port - Ein Port ist der "Kanal" über den eine bestimmte art von internetraffik läuft
    \item IP-Adresse - "eine einzigartige Adresse im Internet", die beim verbinden zum Internet verwiesen wird.  
    \item TCP/IP - Transmission Comunication Protocol/Internet Protocol - Standarte Protokol auf dem das Internet aufbaut, zerklienert Datenpakete dammit sie compact Übertragen werden können.
    \item URL - Uniform Resource Locator - Ein Schema nachdem der Ort einer Recource im Internet angegeben wird
    \item URI - Uniform Resource Identifier - URL sind eine art von URI's, URN sind eine weitere art an URI's die dazu dienen um ein buch Eindeutig zu Identifizieren.  
    \item DNS - Domain Name Server - Übersetzen einen Domain Namen in einer IP Adresse 
    \item CDN - ! Content Delivery Network - 
    \item MME-Type
    \item REST - ! Ein Concept für die abtrennung von HTTP anfragen nutzung
    \item Cookies - Metadaten die Webseiten erstellen um Nutzerdaten, oder Sitzungsdaten lokal im Browser zu speiuchern  
    \item FTP - Ein Protokol für File Sharing.
\end{itemize}

\loesung

\subsection{Aufbeu der URL}
Wie ist eine URL aufgebaut?

\loesung
{Protokol/Service}.{Domain Name}.{Top Level Locator}
\subsection{Pakages}
Definieren Sie Packetorientierte Datenübertragung

\loesung
Die Ideee Daten in kleine "Häpchen" aufzuteilen, verschicken, und dann irgendwo anders zussamen zu bauen.
\subsection{CDN}
Welche Vorteile sind mit CDN verbunden?

\loesung
Phischikalische auswirkung von Distanz auf internet geschwindigkeit zu verringern
\subsection{Webbrowser}
Aus welchen Bestandteilen besteht ein Webbrowser und wie funktioniert er?

\loesung

\subsection{Webserver}
Welche Aufgaben hat ein Weberver?

\loesung

\subsection{Apache 1}
Wie kann man Apache konfigurieren?

\loesung

\subsection{Apache 2}
Was sind Module bei Apache?

\loesung

\subsection{http}
Was ist das http Protokol? Wo im Internet-Protokollstack findet man es wieder? Nutzen sie in Ihrer Erklärung folgende Begriffe:
\begin{itemize}
    \item Status Codes 
    \item Aufbau von Request und Response Nachrichten
    \item Header und Body
    \item http \enquote{Verben}
\end{itemize}

\subsection{Content Type}
Was verbirgt sich hinter dem Content-Type und wie wird er bei http eingesetzt?

\loesung
Um den typ der Resource festzulegen
\subsection{REST}
Definieren und erklären Sie REST.

\loesung

\subsection{http is stateless}
"http is stateless." Erläutern Sie diese Aussage und was folgt daraus allgemein für Webentwicklung?

\loesung
!Es kann keine zustände speichern.
\subsection{Cookies}
Erklären Sie, wie Cookies funktionieren, z. B. von wem werden Cookies erzeugt und abgespeichert und wie die Lebensdauer gesteuert wird, als auch wozu Cookies eingesetzt werden!

\loesung

\subsection{Browser Plugins}
Finden Sie herraus, wie sie mithilfe von Browser-Plugins den Austausch von Daten über das Netzwerk von Webseiten verfolgen und analysieren können.

\loesung

\subsection{Cache}
Was ist Caching, wie funktioniert ein Cache grundsätzlich? Wo im Web kommen Caches zum Einsatz?

\loesung
Caches dienen dazu Daten aus Webseiten die nicht immer neu geliefert werden müssen, lokal zu speichern, und danach beim nächsten Aufruf nur sich die neuen Information zu holen.
\subsection{Web Anwendungen}
Wie sind Webanwendungen typischerweise aufgebaut? Welche Schichten werden in den sogenannten Mehrschichten-Architekturen (bei 2, 3, 4 Schichten) unterschieden? Was ist in diesem Zusammenhang der Unterschied zwischen \enquote{layer} und \enquote{tier}?

\loesung

\subsection{Full-Stack}
Über welche Technologien, sollte ein Full-Stack Developer bescheid wissen?

\loesung

\subsection{Übungsaufgabe 1 - Berufsbild}
Suchen Sie auf Karriereseiten von Unternehmen oder über Jobportale 3 Stellenanzeigen für "full stack"-Webentwickler? Listen Sie auf, welche Kompetenzen und Technologien verlangt werden.

\loesung

\subsection{Übungsaufgabe 2 - HTTP}
Sie rufen mit Ihrem Webbrowser eine Website auf. Welchen Port benutzt der Webserver? Welchen Port benutzt Ihr Browser (Tipp: 80 ist falsch)?

\loesung

\subsection{Übungsaufgabe 3 - HTTP}
Nutzen Sie die \enquote{Entwicklertools} im Webbrowser (empfohlen: Chrome), um sich mit dem http-Protokoll vertraut zu machen. Listen Sie die nützlichsten mit ihrer Funktion auf.

\loesung

\subsection{Übungsaufgabe 4 - HTTP}
Rufen Sie eine bekannte Website auf und untersuchen Sie, was passiert, wenn Sie die Website mehrfach abrufen?

\loesung

\subsection{Übungsaufgabe 5 - HTTP}
Wie kann (vermutlich) die Anzahl von http-Requests verringert werden?

\loesung

\subsection{Übungsaufgabe 6 - HTTP}
Schauen Sie nach, welche Cookies bei Ihnen gesetzt wurden.

\loesung

\subsection{Übungsaufgabe 7 - Software/Tools}
Installieren Sie auf Ihrem eigenen Rechner eine Webserver-Umgebung, idealerweise XAMPP. Testen Sie die Installation.

\loesung

\subsection{Übungsaufgabe 8 - Software/Tools}
Schauen Sie sich an, wie der Webserver konfiguriert werden kann.

\loesung

\subsection{Übungsaufgabe 9 - Software/Tools}
Suchen Sie sich einen Editor, der Sie im Rahmen der weiteren Veranstaltung begleitet. Er sollte mindestens Syntax-Highlighting für HTML, CSS und Javascript bieten. Empfehlung: Atom oder Visual Code oder die Werkzeuge von JetBrains (WebStorm oder PHPStorm)

\loesung

\subsection{Übungsaufgabe 10 - Software/Tools}
Recherchieren Sie nach weiteren geeigneten Browser-Erweiterungen (für Chrome oder Firefox), die bei der Webentwicklung helfen.

\loesung

\subsection{Übungsaufgabe 11 - Software/Tools}
Wenn Sie tiefer in Netzwerktechnologien und Protokolle einsteigen möchten, können Sie das Werkzeug Wireshark nutzen. Achten Sie aber unbedingt auf die legale Nutzung. Sniffing im Hochschulnetz ist untersagt und illegal!

\loesung

\end{document}
